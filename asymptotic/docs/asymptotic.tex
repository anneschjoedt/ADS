\documentclass{tufte-handout}
\usepackage{amsmath}
\usepackage{mathpazo}

\usepackage{tikz}
\usetikzlibrary{chains}
% Define style for nodes
\tikzstyle{every node}=[circle, draw, fill=white,
                        inner sep=0pt, minimum width=9pt,font=\footnotesize\sf]
\tikzstyle{every picture}= [scale=.5,baseline=(current bounding box.center)]
\usetikzlibrary{automata}
\usetikzlibrary{patterns}
\usepackage{booktabs}
\usepackage{siunitx}
\IfFileExists{vc.tex}{\input{vc.tex}}{\newcommand\GITAuthorDate{no git info found}\newcommand\GITAbrHash{-}}  % For version control

\title{Asymptotic Notation}
\author{}
\date {\GITAuthorDate, rev. \GITAbrHash}

\begin{document}
\maketitle
%\section{}
%\subsection{}

\begin{abstract}
\end{abstract}

\section{\textbf{Description}}
  This assignment is about asymptotic notation and growth analysis, with a multiple choice part and a free-text part. 
  
  \medskip Question 1 tests your understanding of Big O and Tilde notation with questions from a previous exam.
  
  \medskip Question 2 forces you to condense what you have learned about asymptotic analysis into few sentences.
  We believe that this can help you in improving  your own understanding as well as train you in a good communication about the topic.
  This question is very much about the journey it takes to formulate a good answer.
  Use it constructively. Try out a few different explanations. Compare with your group members.
  See how your understanding improves as your answer becomes more concise and more precise.

\section{\textbf{Deliverable}}
1. One PDF with your answers, including the name and ITU email of all groups members.

\section{\textbf{Question 1}: from exam 170815 1a, b, c, d, e}
\begin{description}
    \item{a)} Which pair of functions satisfy $f(N)\sim g(N)$?
    \begin{enumerate}
        \item[(A)] $f(N)= 2N$ and $g(N)=N+N+N$ 
        \item[(B)] $f(N)=2N+10 + N^2$ and $g(N)=N^2+10$ 
        \item[(C)] $f(N)=2N^2 + N$ and $g(N)=2N^2 + N^2$ 
        \item[(D)] $f(N)=2N^2 + N$ and $g(N)=N^2+N^3$ 
    \end{enumerate}
\bigskip
    \item{b)} Which pair of functions satisfy $f(N)=O(g(N))$?
      \begin{enumerate}
        \item[(A)] $f(N)=17$ and $g(N)=1$
        \item[(B)] $f(N)=(N+1)\cdot(N+1)$ and $g(N)=N\log N $ 
        \item[(C)] $f(N)=(\log N)\cdot(\log N)$ and $g(N)=2\log N$
        \item[(D)] $f(N)=N^5+5N$ and $g(N)=5N^4$ 

      \end{enumerate}
\bigskip
%\pagebreak
\item{c)} How many stars are printed?
\begin{verbatim}
    for (int i = 1 ; i < N; i = i+2) StdOut.print("**"); 
\end{verbatim}
      \begin{enumerate}
          \item[(A)] $\sim 2\log_2 N$
          \item[(B)] $\sim N$
          \item[(C)] $\sim N\log N$
          \item[(D)] $\sim\frac{1}{2}N^2$
      \end{enumerate}
      \medskip
    
\bigskip\item{d)} How many stars are printed?
         (Choose the smallest correct estimate.)
\begin{verbatim}
    for (int i = 0; i < N/2; i = i+1)
        for (int j = 1; j < N/2; j = 2*j)
            StdOut.print("**");
\end{verbatim}
      \begin{enumerate}
          \item[(A)] $O(\log N)$
          \item[(B)] $O(N)$
          \item[(C)] $O(N\log N)$
          \item[(D)] $O(N^2)$
      \end{enumerate}
\bigskip
\item{e)} What is the asymptotic running time of the following piece
  of code?
         (Choose the smallest correct estimate.)
\begin{verbatim}
    if (N < 1000) for (int i = 0; i < N*N; i = i+1) A[i] = A[i/2];
    else          for (int i = 0; i < N; i = i+1)   A[i] = A[i-1000];
\end{verbatim}
      \begin{enumerate}
          \item[(A)] linear in $N$
          \item[(B)] linearithmic in $N$
          \item[(C)] quadratic in $N$
          \item[(D)] cubic $N$
      \end{enumerate} 
\end{description}

\section{\textbf{Question 2}: In your own words}
\begin{description}

  \medskip\item[(a)]
  In three lines or less, and in your own words, explain why insertion in a heap takes $O(lg N)$ operations.
  \medskip\item[(b)]
  In three lines or less, and in your own words, explain why the asymptotic running time of the "union" operation does not improve from Quick-Find to Quick-Union.
    \medskip\item[(c)]
    In three lines or less, and in your own words, explain why the alterations to Weighted-Quick-Union (from Quick-Union) makes the running time of the "union" operation run in $O(lg N)$.
    \medskip\item[(d)]
    In three lines or less, and in your own words, explain why insertion sort is $O(N^2)$.
    \medskip\item[(e)]
    In three lines or less, and in your own words, explain why insertion sort is not faster than argued for in~(d).
    \footnote{Usually we would phrase this as ``the analysis of insertion sort in (d) is tight''}
    \medskip\item[(f)]
    In three lines or less, and in your own words, explain what asymptotic analysis is (in regards to algorithms).
    \medskip\item[(g)]
    What is the running time of finding a specific element in a heap? What about a binary search tree? In three lines or less, and in your own words, explain what is different between the two.
\end{description}
\end{document}
