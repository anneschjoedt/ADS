\documentclass{tufte-handout}
\usepackage{amsmath}
\usepackage[utf8]{inputenc}
\usepackage{mathpazo}
\usepackage{graphicx}
\usepackage{listings}
\lstset{%
language=bash,%
showstringspaces=false,%
basicstyle=\footnotesize\ttfamily%
}

\usepackage{tikz}
\usetikzlibrary{chains}
\usetikzlibrary{decorations}
\IfFileExists{vc.tex}{\input{vc.tex}}{\newcommand\GITAuthorDate{no git info found}\newcommand\GITAbrHash{-}}  % For version control

\title{Four-Sum}
\author{Holger Dell, Thore Husfeldt, and Riko Jacob}
\date{\GITAuthorDate, rev. \GITAbrHash}
\begin{document}
\maketitle
\begin{abstract}
  Implement two algorithms for the same problem.
  Optionally in two different programming languages.
  Run some experiments.
  You must work in groups of size 3.
\end{abstract}


 \section{Description}

 In the Four-Sum problem, your program is given a list of $N$ (long) integers and it is supposed to detect if there are four of them that sum to $0$.
 For example, the input could be the following file:
 \begin{lstlisting}[language=None]
 10
 -3
 4
 2
 10
 -7
 -12
 9
 13
 0
 -1
 \end{lstlisting}
 The first line is the number $N=10$, then the $N$ (long) integers follow.
 Because $(-3)+10+(-7)+0 = 0$ holds, the correct output is:
 \begin{lstlisting}
 True
 \end{lstlisting}

 Further sample inputs as well as code skeletons can be found at \url{https://bitbucket.org/rikj/bads-labs/downloads/foursum.zip}.

\begin{description}
  \item[Exhaustive search.]
The simplest algorithm performs an exhaustive search using four nested for loops.
Go ahead and implement it in the language that you're comfortable with, Python or Java.
There is a code skeleton called \texttt{sol/Simple.java} or \texttt{sol/simple.py} that you must use.
The program expects a list of $N$ integers on standard input, following the number $N$ itself.
The directory \texttt{data} contains several of such lists for various $N$.

The program must write \texttt{True} to standard output if it finds a solution, and \texttt{False} otherwise;
your program may write other output to standard error (\texttt{stderr}).
Important: Test your program for correctness by running it on various inputs where you know the output, and cross-check between different implementations!

Next, experimentally determine the running time of the program when it is run on your machine on the input files.
The .zip archive contains a handful of inputs for each size.
Report the fastest observed time, the slowest, and the average.
(Try to be half-way serious about this---\emph{e.g.}, restart the computer before the experiment, so as to make sure no spurious processes are running in the background, don't have a World of Warcraft-server or Bitcoin mining programme running at the same time.)
You probably won't be able to handle the larger inputs---don't report experiments that take more than a few minutes.
Plot the result using both a standard and a log-log plot.

\item[\emph{Optional:} Exhaustive search in a different language.]
  Do the same as above, but in the other language.
  This may involve installing a compiler or run-time environment (find the installation instructions here: \url{https://github.itu.dk/pages/algorithms/ads-2020-notes/preparation/software/}), figuring out where the semicolons or brackets go (or don't go), and other annoyances.
  Compare the running times of the two implementations of exhaustive search in the different languages.
  This is the only task that you need to implement in both languages.

\item[Faster.]
  Use the idea of the fast three-sum algorithm in [SW, Chapter 1.4] \texttt{
  ThreeSumFast} to write a faster algorithm for the four-sum problem.
  Your running time must be close to cubic time, proportional to $N^3\operatorname{lg} N$.
  Call it \texttt{Faster.java}  or \texttt{faster.py}.
  Redo the experiments and report them.

  \emph{Optional:} There is a cute, simple, and very satisfying way to solve the problem in time proportional to $N^2\operatorname{lg} N$ instead.
If you can figure that trick out, implement it and hand in that code instead.
\end{description}

\section{Deliverable}
\begin{description}
  \item[Programs.] Implementation of ``Exhaustive search'' and ``Faster'' in your standard programming language (Java or Python~3).
  \item[Optional Program 1.] Implementation of ``Exhaustive search''  in the other programming language (Python~2 is just as good).
  \item[Optional Program 2.] Implementation of your self-designed, even faster algorithm.
  \item[Report.]
    Hand in a \textbf{very brief (max. 2 pages)} report as a PDF, preferably by just finishing the skeleton in \href{https://bitbucket.org/rikj/bads-labs/src/master/foursum/docs/report-foursum.tex}{report-foursum.tex} in the zip-archive. You can edit LaTeX documents on your machine, or using \href{https://www.overleaf.com/}{overleaf.com}. Pay special attention to the figures, they are important.
    Please add all the names and itu-email addresses of the group members as authors in the PDF of the report.
\end{description}
The report and all programs should be submitted on LearnIT as a zip-archive. Note that all necessary code to compile and run your programs, tests, and experiments from the command line must be submitted, and there should be a short \texttt{README.txt} that describes how to run and test your code.

\section{Tips and comments}

\begin{description}
  \item[Java types.] The integers in the input files are very large, and will not fit into the Java data types \texttt{int} or \texttt{Integer}.
    Use \texttt{Long} instead.
    Do not worry about overflow.
  \item[Python dialect.] You can use Python~2, but remember that the \texttt{itu.algs4} library of the course is written in Python~3.
  \item[Course libraries.]
    In this exercise you are \emph{not} required to use the course libraries, say for input/output.
    This is because it is too much overhead to ask students in language $x$ to install the libraries for language $y$.
    Therefore, this particular exercise is very bare-bones.
    You are welcome to use the course library if you want to.
  \item[Measuring time.]
    On a Unix system, the \texttt{time} command gives sufficiently good performance estimates, and is better than sitting with a stopwatch.
    Both Java and Python come with better solutions; the course libraries offer a \texttt{Stopwatch} class, and the Python module \texttt{time} has a function \texttt{time.process\_time()}. You are free to use either. 
    (Note: On many modern Java systems, the compiler performs very funky
    optimisations behind your back, giving wildly divergent measurements for
    the same code on the same machine.  You can handle that by running time
    code a few times before measuring it.  It's a black art.  If you're
    curious, see this note by Peter Sestoft for proper benchmarking in Java:
    \url{https://www.itu.dk/people/sestoft/papers/benchmarking.pdf}.)
  \item[Experiment design.]
    If you like doing repetitive work, you  can run your experiments individually by starting dozens of processes from the command line (or, worse, an IDE) and writing the results into a spreadsheet to compute averages.
    However, such an approach quickly becomes both tedius and error-prone.

    Instead, construct another program that performs all the experiments for you, and writes the results, including a time stamp of the experiment, into another file.
    There are many schools of thought about how to write that program; you can make it part of your original solution, or you can use a separate programme in a dedicated scripting language.
    One useful program (or \texttt{builtin} in \texttt{bash}) is \texttt{time}, a simple usage would be 
    \begin{lstlisting}
      time java Simple < ../data/ints-200-2.txt
    \end{lstlisting}
    Our programs may produce debugging output on \texttt{stderr}, which we need to ignore when collecting experimental results into a file. 
    The following is an example for what this could look like (note that the trailing \verb+\+ continues the command on the next line) (this script \texttt{simpleExp.sh} can be found in the .zip archive:
    \lstinputlisting{../src/simpleExp.sh}

    More advanced solutions (maybe in other languages) could compute averages, or even draw the plots for you.
    One example (that turned out to be somewhat more complicated) can be found in \href{https://bitbucket.org/rikj/bads-labs/src/master/foursum/src/simpleExp.sh}{\texttt{experiment.py}}.
\end{description}

\section{Learning outcomes}

Besides some important secondary skills (input/output, exposing yourself to an unfamiliar programming language, basic experimentation, drawing graphs, computing averages), the main points of this exercise are:
\begin{itemize}
  \item It does \emph{not} make a big difference which language you use.
  \item It \emph{does} make a big difference which algorithm you use.
  \item The slowest observed performance is well characterised by the predicted worst-case running time.
  \item The fastest observed performance contains very little information about general behaviour.
\end{itemize}

\end{document}
